\documentclass[a4paper,11pt]{article}
\usepackage{a4wide}
\usepackage{fullpage}
\usepackage[utf8x]{inputenc}
%\usepackage[slovene]{babel}
%\selectlanguage{slovene}
\usepackage[toc,page]{appendix}
\usepackage[pdftex]{graphicx} 
\usepackage{amsfonts}
\usepackage{amsmath}
\usepackage{setspace}
\usepackage{color}
\definecolor{light-gray}{gray}{0.95}
\usepackage{listings} 
\usepackage{hyperref}
\renewcommand{\baselinestretch}{1.2} 
\renewcommand{\appendixpagename}{Priloge}

\lstset{ 
language=Python,
basicstyle=\footnotesize,
basicstyle=\ttfamily\footnotesize\setstretch{1},
backgroundcolor=\color{light-gray},
}

\title{Topological data analysis \\ Homework 1}
\author{Sara Bizjak (27202020)}
\date{\today}

\begin{document}

\maketitle

\section{Theoretical problems}
\subsection{Exploring different metrics}

%%%%%%%%%%%%%%%%%%%%%%%%%%%%%%%%%%%%%%%%%%%%%%%%%%%%%%%%%%%%%%%%%%%%%%%%%%%%%%%%
\noindent
a) Determing the distances between the points $(2, 1), \ (4, 2), \ (0, 2)$ in metrics $\alpha, \  \beta, \ \gamma$.

\begin{itemize}
    \item Metric $\alpha$.
        
    \begin{equation*}
        \text{d}_{\alpha} \big( (2, 1), \ (4, 2) \big) = \sqrt{2^2 + 1^2} + \sqrt{4^2 + 2^2} = \sqrt{5} + \sqrt{20} = 6.708203932499369.
    \end{equation*}

    \begin{equation*}
        \text{d}_{\alpha} \big( (2, 1), \ (0, 2) \big) = \sqrt{2^2 + 1^2} + \sqrt{0^2 + 2^2} = \sqrt{5} + \sqrt{4} = 4.23606797749979.
    \end{equation*}

    \begin{equation*}
        \text{d}_{\alpha} \big( (4, 2), \ (0, 2) \big) = \sqrt{4^2 + 2^2} + \sqrt{0^2 + 2^2} = \sqrt{20} + \sqrt{4} = 6.47213595499958.
    \end{equation*}

%%%%%%%%%%%%%%%%%%%%%%%%%%%%%%%%%%%%%%%%%

    \item Metric $\beta$.
    
    \begin{equation*}
        \text{d}_{\beta} \big( (2, 1), \ (4, 2) \big) = \sqrt{(2 - 4)^2 + (1 - 2)^2} = \sqrt{4 + 1} = 2.23606797749979.
    \end{equation*}
        
    \begin{equation*}
        \text{d}_{\beta} \big( (2, 1), \ (0, 2) \big) = \sqrt{2^2 + 1^2} + \sqrt{0^2 + 2^2} = \sqrt{5} + \sqrt{4} = 4.23606797749979.
    \end{equation*}
        
    \begin{equation*}
        \text{d}_{\beta} \big( (4, 2), \ (0, 2) \big) = \sqrt{4^2 + 2^2} + \sqrt{0^2 + 2^2} = \sqrt{20} + \sqrt{4} = 6.47213595499958.
    \end{equation*}

%%%%%%%%%%%%%%%%%%%%%%%%%%%%%%%%%%%%%%%%%

    \item Metric $\gamma$.
    
    \begin{equation*}
        \text{d}_{\gamma} \big( (2, 1), \ (4, 2) \big) = |2 - 4| + |1| + |2| = 2 + 1 + 2 = 5.
    \end{equation*}
        
    \begin{equation*}
        \text{d}_{\gamma} \big( (2, 1), \ (0, 2) \big) = |2 - 0| + |1| + |2| = 5.
    \end{equation*}
        
    \begin{equation*}
        \text{d}_{\gamma} \big( (4, 2), \ (0, 2) \big) = |4 - 0| + |2| + |2| = 4 + 2 + 2 = 8.
    \end{equation*}


\end{itemize}

%%%%%%%%%%%%%%%%%%%%%%%%%%%%%%%%%%%%%%%%%%%%%%%%%%%%%%%%%%%%%%%%%%%%%%%%%%%%%%%%
\noindent
b) Draw the open balls $B \big((0, 0), 1 \big)$, $B \big((1, 0), 2 \big)$, $B \big((0, 2), 6 \big)$ in $\alpha$ metric. 
\\
Note that in an open ball are all points that are for $< r$ away from the center, where $r$ is a radius. Since a center is away from itself for $0$, it's always contained in an open ball.
\begin{align*} 
    B \big((0, 0), 1 \big) &= \{(0,0)\} \cup \{(x,y) \in \mathbb{R}^2 : \sqrt{x^2 + y^2} + \sqrt{0^2 + 0^2}< 1 \} 
    \\
    &= \{(0,0)\} \cup \{(x,y) \in \mathbb{R}^2 : \sqrt{x^2 + y^2} < 1 \}. 
\end{align*}

\begin{figure}[ht!]
    \centering
    \includegraphics[width=60mm]{b1.png}
    \caption{$B \big((0, 0), 1 \big)$ in metric $\alpha$.}
\end{figure}

%%%%%%%%%%%%%%%%%%%%%%%%%%%%%%%%%%%%%%%%%%

\begin{align*} 
    B \big((1, 0), 2 \big) &= \{(1,0)\} \cup \{(x,y) \in \mathbb{R}^2 \setminus (1,0) : \sqrt{x^2 + y^2} + \sqrt{1^2 + 0^2} < 2 \}
    \\
    &= \{(1,0)\} \cup \{(x,y) \in \mathbb{R}^2 \setminus (1,0) : \sqrt{x^2 + y^2} < 1 \} .
\end{align*}

\begin{figure}[ht!]
    \centering
    \includegraphics[width=60mm]{b2.png}
    \caption{$B \big((1, 0), 2 \big)$ in metric $\alpha$.}
\end{figure}

 %%%%%%%%%%%%%%%%%%%%%%%%%%%%%%%%%%%%%%%%%

 \begin{align*} 
    B \big((0, 2), 6 \big) &= \{(0,2)\} \cup \{(x,y) \in \mathbb{R}^2 \setminus (0,2) : \sqrt{x^2 + y^2} + \sqrt{0^2 + 2^2} < 6 \} 
    \\
    &= \{(0,2)\} \cup \{(x,y) \in \mathbb{R}^2 \setminus (0,2) : \sqrt{x^2 + y^2} < 4 \}. 
\end{align*}

\begin{figure}[ht!]
    \centering
    \includegraphics[width=60mm]{b3.png}
    \caption{$B \big((0, 2), 6 \big)$ in metric $\alpha$.}
\end{figure}
%%%%%%%%%%%%%%%%%%%%%%%%%%%%%%%%%%%%%%%%%%%%%%%%%%%%%%%%%%%%%%%%%%%%%%%%%%%%%%%%%%%%%%
\noindent
c) Draw the open balls $B \big((0, 0), 1 \big)$, $B \big((1, 0), 2 \big)$, $B \big((2, 2), 3 \sqrt{2} \big)$ in $\beta$ metric. 

\begin{align*} 
    B \big((0, 0), 1 \big) &= \{(x,y) \in \mathbb{R}^2 : \sqrt{(x - 0)^2 + (y - 0)^2} < 1 \}
    \\
    &= \{(x,y) \in \mathbb{R}^2 : \sqrt{x^2 + y^2} < 1 \}.
\end{align*}

\begin{figure}[ht!]
    \centering
    \includegraphics[width=60mm]{b1.png}
    \caption{$B \big((0, 0), 1 \big)$ in metric $\beta$.}
\end{figure}

%%%%%%%%%%%%%%%%%%%%%%%%%%%%%%%%%%%%%%%%%%%%

\begin{align*} 
    B \big((1, 0), 2 \big) &= \{(x,0) \in \mathbb{R}^2 : \sqrt{(x - 1)^2} < 2 \} \cup \{(x, y) \in \mathbb{R}^2, \ y \neq 0 : \sqrt{x^2 + y^2} + \sqrt{1^2} < 2 \}
    \\
    &= \{(x,0) \in \mathbb{R}^2 : |x - 1| < 2 \} \cup \{(x,y) \in \mathbb{R}^2, \ y \neq 0 : \sqrt{x^2 + y^2} < 1 \} 
    \\
    &= \{(x,0) \in \mathbb{R}^2 : -1 < x < 3 \} \cup \{(x,y) \in \mathbb{R}^2, \ y \neq 0 : \sqrt{x^2 + y^2} < 1 \} .
\end{align*}

\begin{figure}[ht!]
    \centering
    \includegraphics[width=60mm]{c2.png}
    \caption{$B \big((1, 0), 2 \big)$ in metric $\beta$.}
\end{figure}

 %%%%%%%%%%%%%%%%%%%%%%%%%%%%%%%%%%%%%%%%%%%

\begin{align*} 
    B \big((2, 2), 3 \sqrt{2} \big) &= \{(x,x) \in \mathbb{R}^2 : \sqrt{2 \cdot (x - 2)^2} < 3 \sqrt{2} \} \cup \{(x, y) \in \mathbb{R}^2, \ x \neq y : \sqrt{x^2 + y^2} + \sqrt{2 \cdot 2^2} < 3 \sqrt{2} \}
    \\
    &= \{(x,x) \in \mathbb{R}^2 : \sqrt{2} \cdot |x - 2| < 3 \sqrt{2} \} \cup \{(x,y) \in \mathbb{R}^2, \ x \neq y : \sqrt{x^2 + y^2} + 2 \sqrt{2} < 3 \sqrt{2} \} 
    \\
    &= \{(x,x) \in \mathbb{R}^2 : -1 < x < 5\} \cup \{(x,y) \in \mathbb{R}^2, \ x \neq y : \sqrt{x^2 + y^2} < \sqrt{2} \} .
\end{align*}

\begin{figure}[ht!]
    \centering
    \includegraphics[width=60mm]{c3.png}
    \caption{$B \big((2, 2), 3 \sqrt{2} \big)$ in metric $\beta$.}
\end{figure}

%%%%%%%%%%%%%%%%%%%%%%%%%%%%%%%%%%%%%%%%%%%%%%%%%%%%%%%%%%%%%%%%%%%%%%%%%%%%%%%%%%%%%%%%%%%%%
\noindent
d) Draw the open balls $B \big((0, 0), 1 \big)$, $B \big((1, 0), 2 \big)$, $B \big((2, 0), 3  \big)$ in $\gamma$ metric. 

\begin{align*} 
    B \big((0, 0), 1 \big) &= \{(0,y) \in \mathbb{R}^2 : |y - 0| < 1 \} \cup \{(x,y) \in \mathbb{R}^2 , \ x \neq 0: |x - 0| + |y - 0| < 1 \}
    \\
    &= \{(0,y) \in \mathbb{R}^2 : -1 < y < 1 \} \cup \{(x,y) \in \mathbb{R}^2 : -1 < x < 1, \ -1 < y < 1 \}.
\end{align*}

\begin{figure}[ht!]
    \centering
    \includegraphics[width=60mm]{d1.png}
    \caption{$B \big((0, 0), 1 \big)$ in metric $\gamma$.}
\end{figure}

%%%%%%%%%%%%%%%%%%%%%%%%%%%%%%%%%%%%%%%%%%%%
\begin{align*} 
    B \big((1, 0), 2 \big) &= \{(1,y) \in \mathbb{R}^2 : |y - 0| < 2 \} \cup \{(x,y) \in \mathbb{R}^2, \ x \neq 1 : |x - 1| + |y - 0| < 2 \}
    \\
    &= \{(1,y) \in \mathbb{R}^2 : -2 < y < 2 \} \cup \{(x,y) \in \mathbb{R}^2 , \ x \neq 1: -1 < x < 3, \ -2 < y < 2 \}.
\end{align*}

\begin{figure}[ht!]
    \centering
    \includegraphics[width=60mm]{d2.png}
    \caption{$B \big((1, 0), 2 \big)$ in metric $\gamma$.}
\end{figure}

 %%%%%%%%%%%%%%%%%%%%%%%%%%%%%%%%%%%%%%%%%%%
\begin{align*} 
    B \big((2, 0), 3 \big) &= \{(2,y) \in \mathbb{R}^2 : |y - 0| < 3 \} \cup \{(x,y) \in \mathbb{R}^2, \ x \neq 2 : |x - 2| + |y - 0| < 3 \}
    \\
    &= \{(2,y) \in \mathbb{R}^2 : -3 < y < 3 \} \cup \{(x,y) \in \mathbb{R}^2 , \ x \neq 2: -1 < x < 5, \ -3< y < 3 \}.
\end{align*}

\begin{figure}[ht!]
    \centering
    \includegraphics[width=60mm]{d3.png}
    \caption{$B \big((2, 0), 3 \big)$ in metric $\gamma$.}
\end{figure}



%%%%%%%%%%%%%%%%%%%%%%%%%%%%%%%%%%%%%%%%%%%%%%%%%%%%%%%%%%%%%%%%%%%%%%%%%%%%%%%%%%%%%%%%%%%%%%%%%%%%%%%%%%%%%%%%%%%%%%%%%%%%%%%%%%%%%%%%%%%%%%%%%%%%%%%%%%%%%%%%%%%%%%%%%%%%%%%%

\newpage
\subsection{Discrete metric}
The discrete metric on a space $X$ is defined as $d: \ X \times X \to \mathbb{R}$, where $d(x, y) = 0$ if $x = y$ and $1$ otherwise.

\noindent
a) Let $X = \mathbb{N}$. Describe $B \big(1, \frac{1}{2} \big)$ and $B \big(2, 1 \big)$.

\begin{center}
$ B \left(1, \frac{1}{2} \right) = \{ x \in \mathbb{N} : d(1, x) < \frac{1}{2} \} = \{ x = 1 : d(1, 1) = 0 \} = \{ 1 \}. $
\\
$ B (2, 1) = \{ x \in \mathbb{N} : d(2, x) < 1 \} = \{ x = 2 : d(2, 2) = 0 \} = \{ 2 \}. $
\\
\end{center}

%%%%%%%%%%%%%%%%%%%%%%%%%%%%%%%%%%%

\noindent
b) The triangle with vertices at distinct integers $a, b, c$ is always equilateral, because the distance between distinct points in discrete metric is always $1$.

%%%%%%%%%%%%%%%%%%%%%%%%%%%%%%%%%%%%%%%%%%%%%%%%%%%%%%%%%%%%%%%%%%%%%%%%%%%%%%%%%%%%%%%%%%%%%%%%%%%%%%%%%%%%%%%%%%%%%%%%%%%%%%%%%%%%%%%%%%%%%%%%%%%%%%%%%%%%%%%%%%%%%%%%%%%%%%%%

\subsection{Homeomorphic spaces}

Let $X = S^{n - 1} \times [0,1] \subset \mathbb{R}^{n+1}$ and $Y = \{ (x_1, \ldots, x_n) \in \mathbb{R}^n \ : \ 1 \leq x_1^2 + \cdots + x_n^2 \leq 4 \}$. What we want to do is to prove that $X$ and $Y$ are homeomorphic.
Therefore, we need to define continuous functions $f: X \to Y$ and $g: Y \to X$ and show that $g = f^{-1}$. We do that by calculating $f \circ g $ and $g \circ f$ and show that they are identities.
\\
To get an idea, we first look at the sketches of spaces $X$ and $Y$ for $n = 1$ and $n = 2$, denoted as $X_1, X_2$ and $Y_1, Y_2$.
\\

\begin{figure}[ht!]
     \begin{minipage}{0.5\textwidth}
         \centering
         \includegraphics[width=50mm]{X_n1.png}
         \caption{$X_1 = S^0 \times [0,1] \subset \mathbb{R}^2$.}
       \end{minipage}\hfill
     \begin{minipage}{0.5\textwidth}
         \centering
         \includegraphics[width=70mm]{Y_n1.png}
         \caption{$ Y_1 = \{ x_1 \in \mathbb{R} : 1 \leq x_1^2 \leq 4 \}$.}
       \end{minipage}\hfill
    \end{figure}

\begin{figure}[ht!]
     \begin{minipage}{0.45\textwidth}
         \centering
         \includegraphics[width=50mm]{X_n2.png}
         \caption{$X_2 = S^1 \times [0,1] \subset \mathbb{R}^3$.}
       \end{minipage}\hfill
     \begin{minipage}{0.55\textwidth}
         \centering
         \includegraphics[width=50mm]{Y_n2.png}
         \caption{$ Y_2 = \{ (x_1, x_2) \in \mathbb{R}^2 : 1 \leq x_1^2 + x_2^2 \leq 4 \}$.}
       \end{minipage}\hfill
    \end{figure}

\newpage
\noindent
We imagine space $X_2$ as an union of disjointed lines connecting upper and lower circle edges. Similarly, we imagine space $Y_2$. Let's present this idea in the following pictures.

\begin{figure}[ht!]
    \begin{minipage}{0.45\textwidth}
        \centering
        \includegraphics[width=60mm]{X_n2_lines.png}
        \caption{$X_2$ as an union of disjoint lines.}
      \end{minipage}\hfill
    \begin{minipage}{0.55\textwidth}
        \centering
        \includegraphics[width=60mm]{Y_n2_lines.png}
        \caption{$Y_2$ as an unoin of disjoint lines.}
      \end{minipage}\hfill
   \end{figure}

\noindent
We see that we can go from space $X_2$ to space $Y_2$ by "flipping" lines down on the plane. And equivalently, from $Y_2$ to $X_2$ we do the same thing but in reverse.
\\
Lines in $X_2$ have the form of $(1 - t) \cdot (x_1, x_2, 0) + t \cdot (x_1, x_2, 1)$ and lines in $Y_2$ have the form of $(1 - t) \cdot (x_1, x_2) + 2  t \cdot (x_1, x_2)$, where $ t \in [0, 1]$.
\\
To get from $X_2$ to $Y_2$ we have to project lines to $x_1 x_2-$plane (by skipping the third coordinate) and then stretch them with factor  $(1 + t)$, where $t \in [0,1]$.
On the other hand, to get from $Y_2$ to $X_2$ we have to squeeze the lines to the unit circle (normalization) and then stretch them up from $0$ to $1$, so we write the new coordinate, denoted by t, as a function of $x_1$ and $x_2$.
Following this idea, functions $f_2: X_2 \to Y_2$ and $g_2: Y_2 \to X_2$ are: \\
$f_2(x_1, x_2, t) = (x_1 \cdot (1 + t), x_2 \cdot (1 + t))$ \ and \
$g_2(x_1, x_2) = \left( \frac{x_1}{\sqrt{x_1^2 + x_2^2}}, \frac{x_2}{\sqrt{x_1^2 + x_2^2}}, \sqrt{x_1^2 + x_2^2} - 1 \right),$  where $t \in [0,1]$.
\\
\\
Let's get back to $X$ and $Y$ spaces and use our idea for general n.
Let $f: X \to Y$ and let $g: Y \to X$. Function $f$ is:

$$ f(x_1, \ldots, x_n, t) =  \left( x_1 \cdot (1 + t), \ldots, x_n \cdot (1 + t) \right) = (1 + t) \cdot (x_1, \ldots, x_n). $$

\noindent
Now, we need to verify that $(1 + t) \cdot (x_1, \ldots, x_n) \in Y$. Because $(x_1, \ldots, x_n, t) \in X$, then $x_1^2 + \cdots + x_n^2 = 1$ and $t \in [0,1]$.
So $((1 + t) \cdot x_1)^2 + \cdots + ((1 + t) \cdot x_2)^2 = (1 + t)^2 \cdot (x_1^2 + \cdots + x_n^2)^2 = (1 + t)^2$ and $1 \leq (1 + t)^2 \leq 4$, so the condition is satisfied.
\\
\\
For function $g$ we need to express the new coordinate $t$ as a function of $(x_1, \ldots, x_n)$. For $t \in [0, 1]$ the relation $t = \sqrt{x_1^2 + \cdots x_n^2} - 1 $ is obvious. 
$$ g(x_1, \ldots, x_n) = \left( \frac{x_1}{ \sqrt{x_1^2 + \cdots + x_n^2}}, \ldots, \frac{x_n}{ \sqrt{x_1^2 + \cdots + x_n^2}}, \sqrt{x_1^2 + \cdots + x_n^2} - 1 \right). $$ 
\noindent
Similarly as we did for function $f$, we now need to verify that \\
$ \left( \frac{x_1}{ \sqrt{x_1^2 + \cdots + x_n^2}}, \ldots, \frac{x_n}{ \sqrt{x_1^2 + \cdots + x_n^2}}, \sqrt{x_1^2 + \cdots + x_n^2} - 1 \right) \in X$.
Because $(x_1, \ldots, x_n) \in Y$ follows  $1 \leq x_1^2 + \cdots + x_n^2 \leq 4$. Since $\left( \frac{x_1}{ \sqrt{x_1^2 + \cdots + x_n^2}} \right)^2 + \cdots + \left( \frac{x_n}{ \sqrt{x_1^2 + \cdots + x_n^2}} \right)^2 = \frac{x_1^2 + \cdots + x_n^2}{x_1^2 + \cdots + x_n^2} = 1$ and $t = \sqrt{x_1^2 + \cdots + x_n^2} - 1$ is equivalent to $ t \in [0, 1]$ , the condition is satisfied.
\\
\\
Clearly, both functions are continuous. \\
All there is left to do is to calculate both compositums. Both compositums are also continuous.

$$ (f \circ g): Y \to X \to Y $$
\begin{align*} 
    (f \circ g)(x_1, \ldots, x_n) &= f \left( \frac{x_1}{ \sqrt{x_1^2 + \cdots + x_n^2}}, \ldots, \frac{x_n}{ \sqrt{x_1^2 + \cdots + x_n^2}}, \sqrt{x_1^2 + \cdots + x_n^2} - 1 \right) \\
    &= \left( \frac{x_1}{ \sqrt{x_1^2 + \cdots + x_n^2}} \cdot \left( \sqrt{x_1^2 + \cdots + x_n^2} \right), \ldots, \frac{x_n}{ \sqrt{x_1^2 + \cdots + x_n^2}} \cdot \left( \sqrt{x_1^2 + \cdots + x_n^2} \right) \right) \\
    &= (x_1, \ldots, x_n) = \text{id}_Y
\end{align*}

$$ (g \circ f): X \to Y \to X$$
\begin{align*} 
    (g \circ f)(x_1, \ldots, x_n, t) &= g \left( x_1 \cdot (1 + t), \ldots, x_n \cdot (1 + t) \right) \\
    &= (1 + t) \cdot \left( \frac{x_1}{ \sqrt{x_1^2 + \cdots + x_n^2}}, \ldots, \frac{x_n}{ \sqrt{x_1^2 + \cdots + x_n^2}}, \sqrt{x_1^2 + \cdots + x_n^2} - 1 \right) \\
    &= (x_1, \ldots, x_n, t) = \text{id}_X
\end{align*}
\noindent 
$ \Longrightarrow X \cong Y $.

\subsection{Homeomorphic spaces}

Let $S_{+}^{n} = \{ (x_1, \ldots , x_{n+1}) \in S^n \ ; \ x_{n + 1} \geq 0\}$ and $B^n = \{ (x_1, \ldots, x_n) \in \mathbb{R}^n \ ; \ x_1^2 + \cdots x_n^2 \leq 1\}$. We want to prove that $S_{+}^{n}$ and $B^n$ are homeomorphic.
To get an idea it's sufficient to look at sketches for $n = 1$.

\begin{figure}[ht]
    \begin{minipage}{0.5\textwidth}
         \centering
         \includegraphics[width=50mm]{X_n1_2.png}
         \caption{$S_{+}^{1} = \{ (x_1, x_2)\in S^1 \ ; \ x_{2} \geq 0\}$.}
    \end{minipage}\hfill
    \begin{minipage}{0.5\textwidth}
         \centering
         \includegraphics[width=70mm]{Y_n1_2.png}
         \caption{$B^1 = \{ x_1 \in \mathbb{R} \ ; \_1^2  \leq 1\}$.}
    \end{minipage}\hfill
\end{figure}
\noindent
To get from left to right, the idea is to project the circle down on the line (so we no longer use coordinate $x_2$ and coordinate $x_1$ stays the same), and from right to left to stretch the line up to circle (we add the new coordinate $x_2$ as a function of $x_1$). 

\begin{figure}[ht]
    \begin{minipage}{0.5\textwidth}
         \centering
         \includegraphics[width=70mm]{XY_2.png}
         \caption{The idea for function $S_{+}^{1} \to B^1$.}
    \end{minipage}\hfill
    \begin{minipage}{0.5\textwidth}
         \centering
         \includegraphics[width=70mm]{YX_2.png}
         \caption{The idea for function $B^1 \to S_{+}^{1}$.}
    \end{minipage}\hfill
\end{figure}
\noindent
Using that idea, we write the regulation for general n.
\\
Let $f: S_{+}^{n} \subset \mathbb{R}^{n + 1} \to B^n \subset \mathbb{R}^n$ and let $g: B^n \in \mathbb{R}^n \to S_{+}^{n} \in \mathbb{R}^{n + 1}$. 
\\
\\
As said before, to determine function $f$, we just skip the last coordinate.
$$ f(x_1, x_2, \ldots, x_{n + 1}) = (x_1, x_2, \ldots, x_n). $$

\noindent
We need to check if $(x_1, \ldots, x_n) \in B^n$. For $(x_1, x_2, \ldots, x_{n + 1}) \in S_{+}^{n}$ the relations $x_1^2 + \cdots + x_n^2 + x_{n + 1}^2 = 1$ and $x_{n+1} \in [0,1]$ are valid and that implies that $x_1^2 + \cdots + x_n^2 = 1 - x_{n+1}^2 \leq 1$, so the condition is satisfied.
\\
\\
For function $g$ we need to express the new coordinate $x_{n + 1}$ as a function of $(x_1, \ldots, x_n)$.
We get $x_{n + 1} = \sqrt{ 1 - x_1^2 - \cdots - x_n^2}$ (we take the positive root because we are in the $S_{+}^{n}$). 

$$ g(x_1, \ldots, x_n) = \left( x_1, \ldots, x_n, \sqrt{ 1 - x_1^2 - \cdots - x_n^2} \right). $$

\noindent
We need to check if $\left( x_1, \ldots, x_n, \sqrt{ 1 - x_1^2 - \cdots - x_n^2} \right) \in S_{+}^{n}$. For $(x_1, \ldots, x_n) \in B^n$ the relation $x_1^2 + \cdots x_n^2 \leq 1 $ is true and $x_1^2 + \cdots + x_n^2 + x_{n + 1}^2 = x_1^2 + \cdots + x_n^2 + 1 - x_1^2 - \cdots - x_n^2 = 1$, so $x_{n+1} \geq 0$ and the condition is satisfied.
\\
Clearly, both function are continuous. 
\\
All there is left to do is to calculate both compositums, which are continuous, too.

$$ (f \circ g): B^n \to S_{+}^{n} \to B^n $$
$$ (f \circ g)(x_1, \ldots, x_n) = f \left(x_1, \ldots, x_n, \sqrt{ 1 - x_1^2 - \cdots - x_n^2} \right) = (x_1, \ldots, x_n) = \text{id}_{B^n}$$
\\
$$ (g \circ f): S_{+}^{n} \to B^n \to S_{+}^{n} $$
$$ (g \circ f)(x_1, \ldots, x_n, x_{n + 1}) = g(x_1, \ldots, x_n) = \left(x_1, \ldots, x_n, \sqrt{ 1 - x_1^2 - \cdots - x_n^2} \right) = (x_1, \ldots, x_n, x_{n + 1}) = \text{id}_{S_{+}^{n}}$$
\noindent
$ \Longrightarrow S_{+}^{n} \cong B^n.$

\section{Programming problems}

\subsection{Deciding connectivity}


Let $G$ be a simple graph with $n$ vertices and $m$ edges.
We write a simple algorithm that returns the connected components of a simple graph. 
All function that we used are in the attached file \texttt{graphcomponents.py}.
\\
Algorithm input: 
\begin{itemize}
    \item a list $ V = [1, 2, \ldots, n]$ of $n$ vertices,
    \item a list of $m$ $2-$tuples $E = [(v_1, v_2), \ldots]$ that represent the $m$ edges.
\end{itemize}
Algorithm output:
\begin{itemize}
    \item a list $[C_1, C_2, \ldots, C_k]$ of all components, where each component $C_1$ is a list of vertices $[v_1, v_2, \ldots, v_k]$
\end{itemize}
We get the connected components of a graph using DFS (Depth Forst Search) algorithm. We could also use BFS algorithm and will get the same result with same time complexity (O$\left( |V| + |E| \right) $).
DFS is an algorithm for graph "research", generally used for connectivity questions -- in our case to determine connected components. 
We first select (any random) vertex to start and then explore as far as possible in a branch and then come back to a fixed point. We keep track if we have visited the vertices connected to it. When we search the graph, we use a stack with LIFO (last in first out) feature.
We also keep a list of all the vertices we have visited since we have to visit each vertex only once.
So we will add a vertex to the stack only if it has not been visited. With visiting a particular vertex, we remove it from the stack. Finally, we'll end up visiting all the vertices and then the stack will be empty.
\\
In the attached code, the input data is pre-processed into a dict (\texttt{makeDictGraph(V, E)}), where the keys are all of the vertices and the values are all vertices that are connected to the key vertex. With that output and dsf algorithm function (\texttt{dfs(graph, start)}) 
we calculate the connected components of a graph in main function \texttt{findComponents(V, E)}.
\\
Let's present the outputs of our algorithm for few examples.
\\
\\
\textbf{Example 1:} \\
Input: \\
\texttt{V = [1,2,3,4,5,6,7,8,9]} \\
\texttt{E = [(1,2),(1,3),(1,8),(3,7),(4,5),(4,6),(4,9),(5,6),(5,9),(7,8)]}
\\
Output: \\
\texttt{[[1, 2, 3, 7, 8], [4, 5, 6, 9]]}
\begin{figure}[ht!]
    \centering
    \includegraphics[width=75mm]{example1.png}
    \caption{Visualization of graph from example 1.}
\end{figure}
\\
\textbf{Example 2:} \\
Input: \\
\texttt{V = [1,2,3,4,5,6,7,8,9,10,11,12]} \\
\texttt{E = [(4,5),(5,4),(1,2),(1,6),(2,1),(2,3),(2,6),(3,2),(3,9),(6,1),(6,2),(6,8),} \\
\texttt{(8,6),(8,9),(9,3),(9,8),(10,11),(11,10),(11,12),(12,11)]}
\\
Output: \\
\texttt{[[1, 2, 3, 6, 8, 9], [4, 5], [7], [10, 11, 12]]}
\begin{figure}[ht!]
    \centering
    \includegraphics[width=75mm]{example2.png}
    \caption{Visualization of graph from example 2.}
\end{figure}
\\
\textbf{Example 3:} \\
Input: \\
\texttt{V = [1,2,3,4,5,6,7,8,9,10,11,12, 13, 14, 15]} \\
\texttt{E = [(1,2),(1,3),(1,4),(1,5),(1,6),(2,1),(2,3),(3,1),(3,2),(3,4),(3,5),(4,1),} \\
\texttt{(4,3),(4,7),(5,1),(5,3),(5,6),(5,7),(6,1),(6,5),(6,7),(7,4),(7,5),(7,6),(8,9),} \\
\texttt{(9,8),(9,10),(9,11),(10,9),(11,9),(12,13),(13,12)]}
\\
Output: \\
\texttt{[[1, 2, 3, 4, 5, 6, 7], [8, 9, 10, 11], [12, 13], [14], [15]]}
\begin{figure}[ht!]
    \centering
    \includegraphics[width=75mm]{example3.png}
    \caption{Visualization of graph from example 3.}
\end{figure}

\subsection{Shelling disks}

Let $P$ be a simple closed polygon. A polygon is a plane figure that is bounded by a finite chain of straight line segments closing 
in a loop to form a closed polygon chain. These segments are called its edges or sides, and the points where two edges meet are the polygon's vertices or corners. 
A polygon is simple if it does not have self-intersections. 
\\
We triangulate $P$, possibly adding vertices in the interior. \\
\\ 
Algorithm input: 
\begin{itemize}
    \item list $T$ of triangles.
\end{itemize}
Algorithm output:
\begin{itemize}
    \item a sequence of all triangles of $P$ such that any initial sequence is homeomorphic to a closed disk (shelling).
\end{itemize}
\noindent
The idea is to write down the triangles as vertices and save their connections. Two triangles are connected if they share an edge.
So the triangle can be connected to maximum three other triangles. If it is connected with less than three, it lays on the edge of the "polygon". 
In that case, we connect the triangle with one extra point outside our polygon.
\\ 
We start making the shelling with any triangle to which we can add any neighbour and sequence will be homeomorphic to a closed disk.
Further, for each triangle $t$ we want to add in our sequence, we check if $G - t$ is connected ($G$ is graph with all triangles). To find the number of connected component, we use the previus task.
If we search through all triangles and the $G - t$ is connected on every step, we find our shelling sequence.

\subsection{Jordan curve theorem}

A simple closed curve in the plane is a connected curve with no self-intersections. 
We will consider a special case of a finite chain of straight line segments closing in a loop to form a simple closed polygonal curve.
The Jordan Curve Theorem states that every simple closed curve in $\mathbb{R}^2$ decomposes $\mathbb{R}^2$  into two components, 
the bounded inside and the unbounded outside.
The main function is in attached file \texttt{jordan.py}.
\\
Algorithm input:
\begin{itemize}
    \item the curve $P = [(x_1, y_1), \ldots, (x_n, y_n)]$, where $(x_i, y_i)$ are vertices,
    \item the point $T = (x_0, y_0)$.
\end{itemize}
Algorithm output:
\begin{itemize}
    \item True if the point $T$ lies inside the polygon $P$ and False otherwise.
\end{itemize}

\noindent
The idea here is to count how many times a line from the point to infinity (in any direction) crosses any edge of the polygon. For example, if the point is in the polygon, the line from that point will have to leave the polygon by crossing some edge.
It sure can re-enter the polygon, but it always has to leave again, making the number of crossing \textit{uneven}. On the other hand, if the number of crossings is \textit{even}, the point is always ouside the polygon.
\\
To solve this task I used a function \texttt{inside} from python library \texttt{shapely}, that directly checks if point $T$ lies in polygon $P$. 
\\
In the same file I have another function  -- the algorithm checks every edge of the polygon to determine if the ray from the point crosses it. The runtime of the algorithm is $O(|P |)$, where $|P |$ is number of the edges in polygon $P$.
\\
\\
\noindent
\textbf{Example 1:} \\
Input: \\
\texttt{P = [(0,0),(3,0),(3,3),(6,3),(6,7),(0,7)]}
\\
\texttt{T = (1, 3)}
\\
Output: \\
\texttt{True}
\begin{figure}[ht!]
    \centering
    \includegraphics[width=70mm]{j1.png}
    \caption{Visualization of polygon and point from example 1.}
\end{figure}


\noindent
\textbf{Example 2:} \\
Input: \\
\texttt{P = [(-2,5),(5,5),(5,3),(3,2),(3,-1),(-1,1),(-4,4)]}
\\
\texttt{T = (-6, 3)}
\\
Output: \\
\texttt{False}
\begin{figure}[ht!]
    \centering
    \includegraphics[width=100mm]{j2.png}
    \caption{Visualization of polygon and point from example 2.}
\end{figure}


\end{document}